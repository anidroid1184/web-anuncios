\documentclass[11pt,a4paper]{article}
\usepackage[utf8]{inputenc}
\usepackage[spanish]{babel}
\usepackage{geometry}
\geometry{a4paper, margin=2.5cm}
\usepackage{xcolor}
\usepackage{booktabs}
\usepackage{graphicx}
\usepackage{hyperref}
\usepackage{fancyhdr}

% Colores corporativos
\definecolor{primary}{RGB}{0, 102, 204}
\definecolor{secondary}{RGB}{102, 102, 102}
\definecolor{success}{RGB}{76, 175, 80}
\definecolor{warning}{RGB}{255, 152, 0}

\pagestyle{fancy}
\fancyhf{}
\rhead{Análisis de Campaña Publicitaria}
\lhead{\today}
\rfoot{Página \thepage}

\title{Análisis Forense de Campaña Publicitaria}
\author{Departamento de Marketing Analytics}
\date{\today}

\begin{document}

% Portada
\maketitle
\thispagestyle{empty}
\newpage

% Índice de contenidos
\tableofcontents
\newpage

% Resumen Ejecutivo
\section*{Resumen Ejecutivo}
\addcontentsline{toc}{section}{Resumen Ejecutivo}
Este reporte presenta un análisis detallado de la reciente campaña publicitaria, evaluando aspectos clave como la composición visual, redacción de mensajes, psicología del público objetivo, optimización móvil y potencial de conversión. Se analizaron un total de 12 anuncios, con un enfoque en mejorar la efectividad de futuras campañas.

\newpage

% Análisis Detallado por Anuncio
\section{Análisis Detallado por Anuncio}
\addcontentsline{toc}{section}{Análisis Detallado por Anuncio}
\begin{table}[h]
    \centering
    \caption{Evaluación de Anuncios}
    \begin{tabular}{lcccccc}
        \toprule
        \textbf{ID del Anuncio} & \textbf{Composición Visual} & \textbf{Redacción} & \textbf{Psicología} & \textbf{Optimización Móvil} & \textbf{Potencial de Conversión} \\
        \midrule
        829095826733139 & 8 & 7 & 8 & 9 & 8 \\
        1321488129781413 & 7 & 6 & 7 & 8 & 7 \\
        1093933469480670 & 6 & 5 & 6 & 7 & 5 \\
        1287206829872336 & 7 & 6 & 7 & 8 & 6 \\
        2093318641406565 & 9 & 8 & 9 & 9 & 9 \\
        2695670124099200 & 6 & 5 & 5 & 7 & 5 \\
        \bottomrule
    \end{tabular}
\end{table}

\newpage

% Comparación de Formatos
\section{Comparación de Formatos}
\addcontentsline{toc}{section}{Comparación de Formatos}
La campaña incluyó 30 imágenes y 2 videos. Se observó que los anuncios con imágenes tuvieron un rendimiento más consistente en términos de claridad visual y optimización móvil, mientras que los videos ofrecieron un mayor potencial de engagement debido a su naturaleza dinámica.

\newpage

% Top 3 y Bottom 3 Anuncios
\section{Top 3 y Bottom 3 Anuncios}
\addcontentsline{toc}{section}{Top 3 y Bottom 3 Anuncios}
\subsection*{Top 3 Anuncios}
\begin{itemize}
    \item \textbf{2093318641406565}: Excelente uso de colores y jerarquía visual clara. Alta efectividad emocional.
    \item \textbf{829095826733139}: Colores vibrantes y mensaje claro, aunque el CTA podría ser más urgente.
    \item \textbf{1321488129781413}: Buena legibilidad y alineación moderada con el público objetivo.
\end{itemize}

\subsection*{Bottom 3 Anuncios}
\begin{itemize}
    \item \textbf{2695670124099200}: Colores apagados y mensaje débil con bajo engagement esperado.
    \item \textbf{1093933469480670}: Elementos visuales básicos y mensaje poco claro.
    \item \textbf{1287206829872336}: Falta de jerarquía visual clara y beneficios no suficientemente atractivos.
\end{itemize}

\newpage

% Patrones Identificados
\section{Patrones Identificados}
\addcontentsline{toc}{section}{Patrones Identificados}
Se identificaron patrones comunes en los anuncios más exitosos, como el uso efectivo de colores contrastantes, mensajes claros con CTAs fuertes y una optimización móvil adecuada. Los anuncios menos efectivos carecían de estos elementos clave.

\newpage

% Recomendaciones Estratégicas
\section{Recomendaciones Estratégicas}
\addcontentsline{toc}{section}{Recomendaciones Estratégicas}
\begin{itemize}
    \item \textbf{Visuales}: Mejorar el contraste de colores y la jerarquía visual para aumentar el engagement.
    \item \textbf{Copywriting}: Fortalecer los CTAs y clarificar los beneficios para mejorar el mensaje.
    \item \textbf{Targeting}: Utilizar desencadenantes emocionales más fuertes para conectar con la audiencia.
    \item \textbf{Móvil}: Asegurar que todos los elementos sean amigables para el pulgar y legibles en dispositivos móviles.
\end{itemize}

\newpage

% Conclusiones
\section{Conclusiones}
\addcontentsline{toc}{section}{Conclusiones}
El análisis de la campaña publicitaria revela áreas clave para mejorar, especialmente en la optimización de mensajes y la composición visual. Al implementar las recomendaciones estratégicas, se espera un aumento significativo en el rendimiento de futuras campañas.

\end{document}